\newpage
\section{Introduction}
\label{sec:introduction}

%Il n'y a pas d'espace au d�but du paragraphe.
\paragraph{}Dans le cadre du module de de Developpement Web du second Semestre de L2, les �tudiants doivent r�alis� en bin�me un projet en r�utilisant les �l�ments appris en cours. 
Le projet consiste en la r�alisation d'un site web permettant la mise en valeur et la recherche dans les donn�es publiques de l'ONISEP, concernant les �tablissements d'enseignements sup�rieur en France.
Notre bin�me est compos� de Matthieu VILAIN, �tudiant en L2 CMI SIC, et de Quentin GERARD �tudiant en L2 CMI SIC.

\subsection{Cahier des charges}

\paragraph{Objectif}{}
Cr�er un site web agr�gateur de donn�es sur les �tablissement sup�rieur fran�ais avec recherche et tri multi-crit�res.

\paragraph{Contraintes}{}
\begin{itemize}
 \item Utiliser les donn�es des fichiers .csv ou .xml
 \item Mettre en place une recherche de donn�es
 \item Mettre en place un syst�me de tri des donn�es
 \item Afficher des statistiques sur les donn�es sous forme de graphique
 \item Mettre en place un syst�me d'historique en utilisant les cookies
\end{itemize}
Le site devra �tre disponible sur le serveur web de l'universit�. 
Il devra �galement �tre valide HTML5 et CSS3.

\paragraph{Acc�s au site}
Vous pouvez acceder au site en suivant l'un de ces deux liens :
\begin{itemize}
 \item http://projetwebmg.16mb.com
 \item https://10.40.128.22/~qgerard/projetWeb
\end{itemize}
